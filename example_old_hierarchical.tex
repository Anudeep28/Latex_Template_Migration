\documentclass[12pt]{article}
\usepackage[utf8]{inputenc}

\title{Research Paper with Hierarchical Structure}
\author{Author Name}

\begin{document}

\maketitle

\section{Introduction}
This introduces the research problem and provides context.

\section{Model Description}
We propose a novel model architecture with the following components.

\subsection{Model Components}
The model consists of three main components that work together.

\subsubsection{Encoder Component}
The encoder processes input data using multiple layers.
It applies attention mechanisms to capture dependencies.

\paragraph{Attention Mechanism}
We use multi-head self-attention with 8 heads.
The attention mechanism allows the model to focus on relevant parts.

\paragraph{Layer Normalization}
Each layer includes batch normalization for stability.
This improves training convergence significantly.

\subsubsection{Decoder Component}
The decoder generates output from encoded representations.
It uses cross-attention to attend to encoder outputs.

\subsection{Model Architecture}
The overall architecture follows a transformer-based design.
We stack 6 encoder and 6 decoder layers.

\paragraph{Positional Encoding}
Sinusoidal positional encodings are added to input embeddings.
This allows the model to utilize sequence order information.

\subsection{Training Strategy}
We employ a two-stage training approach for optimal results.

\subsubsection{Pre-training Phase}
The model is first pre-trained on a large corpus.
We use masked language modeling as the objective.

\subsubsection{Fine-tuning Phase}
Fine-tuning is performed on task-specific data.
Learning rate is reduced by a factor of 10.

\section{Experiments}
We conducted extensive experiments to evaluate performance.

\subsection{Datasets}
Three benchmark datasets were used for evaluation.

\subsection{Evaluation Metrics}
We report accuracy, F1-score, and inference time.

\section{Results}
Our model achieves state-of-the-art performance.

\subsection{Quantitative Results}
The model achieves 95\% accuracy on the test set.

\subsection{Qualitative Analysis}
Case studies demonstrate the model's reasoning capabilities.

\section{Conclusion}
We presented a novel approach with strong empirical results.

\end{document}
